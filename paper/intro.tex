\documentclass[main.tex]{subfiles}

\begin{document}

Euclidean rhythms are a large class of rhythms which can be generated through the Euclidean Algorithm. Making its first appearence in Euclid’s Element, it’s one of the oldest algorithm, and it was used to compute the greatest common divisor of two given integers. 
It was Godfried Toussaint to discover, in 2005, that the Euclidean algorithm’s structure may be used to generate the aforementioned class of rhythms. Moreover, he found that those patterns are often present traditional and world music\cite{Toussaint:2004:euclidean:rhythm}. He noticed that the main characteristic of those rhythms is that their onset pattern are distribuited as evenly as possible.
Since then, Euclidean rhythm generators started to spread in the music production world, since with very few parameters one can generate very complex patterns and polyrhythms. A lot of eurorack synthesizer modules have been developed: Mutable Instruments Yarns\cite{MutableInstruments:Yarns}, vpme.de Euclidean Circles\cite{vpme.de} and 2HP Euclid\cite{2hp:Euclid}, just to name a few. 
This paper will present a software implementation of a drum machine capable of generating Euclidean rhythms, using Supercollider programming language.

\end{document}