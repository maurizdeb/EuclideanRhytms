\documentclass[main.tex]{subfiles}

\begin{document}
The purpose of the researchers in the past year was to find a easy way to generate all the possible rhythms present in music, in order to extend the dictionary of Automatic Music Composition\cite{Allouche:2002:AutomSeq:book}.
The derivation of G. Touissant\cite{Toussaint:2004:euclidean:rhythm} comes from an analyisis over Bjorklund's studies on SNS accelerators\cite{Bjorklund:2004:euclidean:rhythm} in nuclear physics. In this case, time is divided into intervals and during some of these intervals a gate is to be enabled by a timing system that generates pulses that accomplish this task. The problem for a given number \textit{n} of time intervals, and another given number \begin{math}k<n\end{math} of pulses, is to distribute the pulses as evenly as possible among these intervals.\\
In our case the algorithm to be develped take as input a tuple \textit{(k, n)}, where \textit{n} represents the length of the played sequence and \textit{k} the number of onsets inside the sequence. The output is a sequence of length \textit{n}, where the \textit{k} onsets are equally spaced inside the sequence. For example, if we consider the tuple \textit{(4, 16)}, the result has to be a sequence like:\\
[2mm]
[1, 0, 0, 0, 1, 0, 0, 0, 1, 0, 0, 0, 1, 0, 0, 0]
\\
\end{document}